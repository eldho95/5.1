\section{Background Literature}~\label{sec:literature}
The aim of trajectory k-anonymity is to conceal a single trajectory within a group of at least k-1 other trajectories. The assumption is that any trajectory in the published dataset should be indistinguishable from k-1 other trajectories, making it impossible for an adversary to locate the individual behind a trajectory with a probability greater than 1/k.Suppression of attribute values is a technique for achieving k-anonymity that is commonly used on discrete and/or semantic data where perturbation methods are ineffective. One of the first suppression-based approaches for anonymizing trajectory data. Trajectories are described as sequences of addresses drawn from an address domain P. The adversary has power over subsets of P addresses, so his information is defined as representations of initial trajectories over the addresses in P that he knows about.Terrovitis and Mamoulis\cite{terrovitis2008privacy} suggest a greedy algorithm that ensures that no address unknown to the adversary can be connected to a recipient with a probability greater than a given threshold. Instead of generalisation, Domingo-Ferrer suggested a new method based on microaggretion and permutation. They begin by introducing a novel distance metric that takes into account both the spatial and temporal dimensions of trajectories. The distance measure is versatile enough to be applied to both spatiotemporal and time series results. The authors suggest that based on this distance metric, trajectories be clustered to reduce intra-cluster distance. Individuals' privacy is clearly jeopardised if their trajectories are revealed in a manner that causes the individual behind the trajectory to be re-identified. We add to the literature on privacy-preserving trajectories by providing a distance measure for trajectories that naturally considers both spatial and temporal dimensions of trajectories, is polynomial time computable, and can cluster trajectories not specified over the same time period \cite{domingo2012microaggregation}. Other current similarity metrics for trajectories that are suitable for anonymization purposes can be naturally instantiated using our distance scale.