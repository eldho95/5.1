\section*{Abstract}
Human trajectories are often obtained and used for experimental and commercial purposes. However, releasing trajectory data without careful handling could result in significant privacy breaches. A vast body of work is devoted to blending one's trajectory with that of others in order to prevent any particular trajectory from being re-identified. However, their implementations are insufficiently secure because they do not deter semantic attacks, which ensures that attackers can obtain private information about individuals by exploiting semantic features of commonly visited locations in a trajectory without re-identification \cite{tu2017beyond}. We propose a new distance metric for trajectories that accounts for both clustering and obfuscation in the microaggregation method. We suggest a trajectory anonymisation heuristic approach based on this distance calculation that ensures each trajectory is indistinguishable from k-1 other trajectories. Wireless connections can be formed from almost every habitable location on the planet, resulting in a myriad of connection-based monitoring mechanisms including GPS, GSM, and RFID. As a result, everyday trajectories reflecting people's action are being collected and analysed. A trajectory, on the other hand, could include confidential and private information, raising the question of whether spatio-temporal data can be released in a secure manner. K-anonymity has gained increased interest and has been widely analysed in various ways among the various solution methods that have been suggested to address this issue \cite{gkoulalas2010providing}. New types of data, such as location data capturing user activity, pave the way for cutting-edge applications, such as the already available Location Based Services (LBSs). Given that these providers presume in-depth knowledge of smartphone users' whereabouts, it's safe to assume that the inferred knowledge would compromise the users' privacy . As a result, concrete methods are needed to protect the identity of smartphone users when making requests.