\section{The Fr´{e}chet/Manhattan distance and the
trajectory anonymisation problem}~\label{sec:design}
In microaggregation, the option of distance metric is crucial. It has an effect on how trajectories are grouped and, in most cases, on the anonymization method. A trajectory distance measure can be described by a variety of variables. A distance metric, for example, may only consider trajectories within a given timeframe, or it might look for spatial similarities regardless of direction or sampling rate, or it might take into account trajectory characteristics like speed and angle. The Fr'{e}chet distance is a loosely based distance metric that we suggest in this paper. The Fr'{e}chet distance, also known as the dog-leash distance, is based on the assumption that a human walks along one trajectory and his dog walks over the other. Both can fly at a positive yet independent pace. The Fr'{e}chet distance determines the minimum length of leash allowed to walk the person's dog. The shorter the leash, the tighter the two curves seem to be. The anonymization approach suggested in this article is based on k-microaggregation, which is a method for anonymizing clusters with at least k homogeneous trajectories separately. Clustering and the obfuscation method are the two key components of our microaggregation-based strategy. To solve the above-mentioned k-microaggregation problem, we take a greedy approach. A pivot trajectory is used to describe each cluster, and each cluster includes k-1 other trajectories that are similar to the pivot trajectory. In other words, we use the number of squared distances between the pivot trajectory and the other trajectories in the cluster as a homogeneity criterion.The clustering technique and the Fr'{e}chet/Manhattan coupling distance mentioned above are used to provide a privacy-preserving approach for publishing trajectories. About the fact that the coupling distance works well for trajectories recorded at various sampling rates, the lower the sampling rate, the closer it approaches the classical Fr'{e}chet distance.

\subsection{Overview of Dataset}

We use a synthetic dataset developed with Brinkhoff's system , which is often used to test privacy-preserving methods, to compare our anonymisation process with other approaches. The synthetic data generated by Brinkhoff's generator has the benefit of being easily transferrable and reproducible. As a result, we'll go through the parameters that were used to create the dataset of trajectories that we used in our experiments \cite{brinkhoff2002framework}.Over the map of Oldenburg, the following generation parameters were used: 6 moving object classes and 3 external object classes; 5 moving objects and 3 external objects were produced per time-stamp; the overall lifetime of a trajectory was set to 1,000 time-stamps; speed was set to ten; and report probability was set to one thousand. Brinkhoff's generator produced 5, 000 synthetic trajectories, with a total of 492, 105 locations in the German city of Oldenburg and an average of 98.421 locations per trajectory.

\section{Methodology}
It focuses primarily on questions that are used to generate aggregate statistics.
This type of query is usually measurable, so it can be characterised as a function on the domain of all spatio-temporal databases that span a metric space.The method is opposed to approaches focused on generalisation and permutation. The distance threshold in the generalisation-based approach requires the Log-cost distance calculation to discard outlier locations. We set the distance threshold to its highest value since we only consider noiseless synthetic data in this section. Instead, the permutation-based approach considers both a distance and a time threshold to discard outlier positions during the obfuscation phase.\\
The spatiotemporal range query is a well-known form of observable query in trajectory analysis. Sometime Definitely Within ( is true if and only if there exists a moment when trajectory T is inside region R ) and Always Definitely Within ( is true if and only if trajectory T is within region R at all times. ) are two such queries.To create spatial-space queries, we considered regions whose radius is distributed randomly over the natural number range [0, 500]. This interval's limit is just a fraction of the total duration of each trajectory, which is 7284. Keep in mind that the smaller the spatial interval, the narrower the spatial-range demand becomes, and the more difficult it becomes for an anonymization strategy to add spatial distortion without lowering utility.For any cluster size and time period, the system outperforms the alternatives. When the level of privacy given rises, so does the change in terms of utility. Our method performs marginally better than the generalisation-based approach for k = 2, but substantially better for k 4, 8 than the generalisation-based approach. This ensures that our method effectively groups and anonymizes trajectories.