\section{Methodology}
It focuses primarily on questions that are used to generate aggregate statistics.
This type of query is usually measurable, so it can be characterised as a function on the domain of all spatio-temporal databases that span a metric space.The method is opposed to approaches focused on generalisation and permutation. The distance threshold in the generalisation-based approach requires the Log-cost distance calculation to discard outlier locations. We set the distance threshold to its highest value since we only consider noiseless synthetic data in this section. Instead, the permutation-based approach considers both a distance and a time threshold to discard outlier positions during the obfuscation phase.\\
The spatiotemporal range query is a well-known form of observable query in trajectory analysis. Sometime Definitely Within ( is true if and only if there exists a moment when trajectory T is inside region R ) and Always Definitely Within ( is true if and only if trajectory T is within region R at all times. ) are two such queries.To create spatial-space queries, we considered regions whose radius is distributed randomly over the natural number range [0, 500]. This interval's limit is just a fraction of the total duration of each trajectory, which is 7284. Keep in mind that the smaller the spatial interval, the narrower the spatial-range demand becomes, and the more difficult it becomes for an anonymization strategy to add spatial distortion without lowering utility.For any cluster size and time period, the system outperforms the alternatives. When the level of privacy given rises, so does the change in terms of utility. Our method performs marginally better than the generalisation-based approach for k = 2, but substantially better for k 4, 8 than the generalisation-based approach. This ensures that our method effectively groups and anonymizes trajectories.