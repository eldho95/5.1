%========= Introduction
\section{Introduction}~\label{sec:introduction}
The strategy of trajectory k-anonymity is widely used to preserve trajectory privacy. Visual identity was once the only method of gathering spatio-temporal data from individuals. This job is much simpler nowadays since no direct human interference is needed for monitoring and recording. Surveillance cameras, social networks, credit card transfers, and a slew of other widely used devices and utilities capture this information automatically. The widespread use of location-aware devices such as cell phones and GPS receivers today makes it much easier for businesses and policymakers to gather massive amounts of data about people's movements.Despite all of these advantages, there are clear risks to people's privacy if their activity data is released in a manner that causes the individual behind a trajectory to be re-identified. Simply looking at the places reached by a trajectory will expose personal details such as religious, political, or sexual orientation. As time information reveals user preferences, it can be used for illegal advertising and user profiling, posing a privacy risk \cite{yarovoy2009anonymizing}. As a result, simply knowing where a person has been might be enough to classify his path in a database. Consider the case of a GPS programme that tracks citizen trajectories. An early morning trajectory is likely to begin at the user's home and end at the user's office, according to their daily routine. This basic observation may be enough to re-identify a user's path with accuracy. Technologies capture massive volumes of activity data automatically. These data must be made public in order to boost transportation, consider regional economic conditions, and so on. Individuals' anonymity is clearly jeopardised if their trajectories are revealed in a manner that requires re-identification of the person behind the trajectory \cite{domingo2010privacy}. An appropriate approach for obtaining anonymity is de-identification, which entails removing identifying features from individuals. However, other types of attributes known as quasi-identifiers, which are non-identifying attributes that, when paired with external information, will uniquely identify the individual behind a record, are often insufficient to preserve privacy. Regrettably, any location may be called a quasi-identifier in the case of spatial-temporal results. Methods for obtaining k-anonymity on microdata can't be applied to spatiotemporal data specifically. \cite{sweeney2002k}\\
For trajectories that are well matched to clustering and obfuscation, a distance metric is suggested. Since the distance is roughly dependent on the Fr'{e}chet distance, it can be computed quickly. The Fréchet distance is a natural and intuitive way to compute the resemblance of two (polygonal) curves, which is a common operation. While a simple algorithm computes it in near-quadratic time, a highly subquadratic algorithm is impossible to find unless the Strong Exponential Time Hypothesis is proven false. Nonetheless, quick and functional Fréchet distance implementations, particularly for realistic input curves, are highly desirable \cite{sharma2019map}. The new method has many advantages: (a) it can handle non-overlapping trajectories; (b) it generates a number of corresponding points in addition to a distance value, which can be used later in the obfuscation process; and (c) it takes into account the form of the trajectories due to the Fr'{e}chet distance's existence.
